\documentclass[a4wide]{scrartcl}

\usepackage[T1]{fontenc}
\usepackage[utf8]{inputenc}
\usepackage{amssymb}

\author{Simon Stähler, Martin van Driel}
\title{Kerner manual}


\begin{document}
 \maketitle
 
 \begin{abstract}
  This document is a very preliminary manual how to use the kerner program together with AxiSEM. It will be extended (hopefully).
 \end{abstract} 
 
 \section{Prequisites}
 The \textit{kerner} should be run under Linux in combination with a recent version of AxiSEM (Release version 1.1 is not enough!). To get the latest version of AxiSEM, use \\
 \verb|git clone https://github.com/geodynamics/axisem.git|.
 
 \subsection{NetCDF}
 NetCDF is needed as well, preferrably version 4.2. It can be downloaded and installed using the script shipped with AxiSEM.
 The NetCDF libraries in Ubuntu 14.04 are enough
 
 \subsection{Compiler}
 Gfortran >4.6

 \subsection{Run AxiSEM}
 The kerner needs two separate AxiSEM runs, one for the forward and one for the backward wavefield. One has to decide for a frequency first. This should especially depend on the disk space. A 50s run needs 5 GB in total, the space scales with the cube of the frequency. The background model of the mesh does not matter.
 \subsubsection{Forward field}
 Set these parameters, all others do not matter: \\
 inparam\_basic
 \begin{verbatim}
  SIMULATION_TYPE   	moment
  SEISMOGRAM_LENGTH  	1800.
  MESHNAME            	IASP_50s
  LAT_HETEROGENEITY   	false
  SAVE_SNAPSHOTS       	false
 \end{verbatim}
 inparam\_advanced
 \begin{verbatim}
  SAMPLING_PERIOD     0.0
  SOURCE_PERIOD       0.0
  SOURCE_FUNCTION     errorf
  USE_NETCDF          true
  KERNEL_WAVEFIELDS   true
  
  KERNEL_DUMPTYPE     displ_only
  KERNEL_SPP          8
  KERNEL_SOURCE       igno
  
  KERNEL_COLAT_MIN   00.
  KERNEL_COLAT_MAX   180.

  KERNEL_RMIN        0000.
  KERNEL_RMAX        6372.
 \end{verbatim}
 Set the requested depth in CMTSOLUTION 
 
 \subsubsection{Backward field}
 Set these parameters, all others do not matter:  \\
 inparam\_basic:
 \begin{verbatim}
  SIMULATION_TYPE     	force 
  SEISMOGRAM_LENGTH  	1800.
  MESHNAME            	IASP_50s
  LAT_HETEROGENEITY   	false
  SAVE_SNAPSHOTS       	false
 \end{verbatim}
 inparam\_advanced:
 \begin{verbatim}
  SAMPLING_PERIOD     0.0
  SOURCE_PERIOD       0.0
  SOURCE_FUNCTION     errorf
  USE_NETCDF          true
  KERNEL_WAVEFIELDS   true
  
  KERNEL_DUMPTYPE     displ_only
  KERNEL_SPP          8
  KERNEL_SOURCE       igno
  
  KERNEL_COLAT_MIN   00.
  KERNEL_COLAT_MAX   180.

  KERNEL_RMIN        0000.
  KERNEL_RMAX        6372.
 \end{verbatim}
 \subsection{Reorder fields in NetCDF file}
 After the runs have finished, run the script \verb|field_transform.csh| in the run directories. You may have to adapt the directory names of the forward and backward runs. This script will occupy another several GB on your disk. 
 
 \section{Run the kerner}
 The kerner settings are mainly set in \verb|inparam_basic|. The variables \verb|fwd_dir| and \verb|bwd_dir| should point to the run directories of the AxiSEM runs. I always use symbolic links in the \verb|wavefield| directory.\\
 The kerner itself is run with:
 \begin{verbatim}
  mpirun -n $NTASKS ./kerner $PATH_TO_inparam_basic
 \end{verbatim}
 It can be run from any directory, as long as the \verb|$PATH_TO_inparam_basic| and the directories set within are correct. \verb|$NTASKS| has to be at least 2, to have one master and one slave.\\
 For just plotting wavefields, change the setting \verb|WHAT_TO_DO| in \verb|inparam_basic| and start the code without MPI.
 
 
\end{document}
